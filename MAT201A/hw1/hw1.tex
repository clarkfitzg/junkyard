% A simple template for LaTeX documents

\documentclass[12pt]{article}

% Begin paragraphs with new line
\usepackage{parskip}  

% Change margin size
\usepackage[margin=1in]{geometry}   

% Supports proof environment
\usepackage{amsthm}

% Allows writing \implies and align*
\usepackage{amsmath}

% Allows mathbb{R}
\usepackage{amsfonts}

%%%%%%%%%%%%%%%%%%%%%%%%%%%%%%%%%%%%%%%%%%%%%%%%%%%%%%%%%%%%

\title{Math 201A - HW 1}
\date{13 October 2014}
\author{Richard Clark Fitzgerald}

\begin{document}

\maketitle

\textbf{Problem 1.}
Let $(X, d)$ be a metric space, and let $x, y, w, z \in X$.

\textbf{a)}
Prove that
\[
    d(x, y) \geq |d(x, z) - d(z, y)|.
\]

\begin{proof}

Applying the triangle inequality two times we have
\begin{align*}
    d(x, z) \leq d(x, y) + d(y, z) 
    &\implies 
    d(x, z) - d(y, z) \leq d(x, y) \\
    d(y, z) \leq d(y, x) + d(x, z)
    &\implies 
    d(y, z) - d(x, z) \leq d(y, x) = d(x, y).
\end{align*}
For $a, b \in \mathbb{R}$ if $a \leq b$ and $-a \leq b$, then
$|a| \leq b$. We have shown this, so we have the result:
\[
    d(x, y) \geq |d(x, z) - d(z, y)|.
\]

\end{proof}

\textbf{b)}
Prove that
\[
    d(x, y) + d(z, w) \geq |d(x, z) - d(y, w)|.
\]

\begin{proof}

By part a) we have
\begin{align*}
   d(x, y) &\geq |d(x, z) - d(z, y)| \\
   d(z, w) &\geq |d(z, y) - d(y, w)|.
\end{align*}
Adding both sides of the above inequalities produces
\begin{align*}
   d(x, y) + d(z, w) &\geq |d(z, y) - d(y, w)| + |d(x, z) - d(z, y)| \\
                     &\geq |d(z, y) - d(y, w) + d(x, z) - d(z, y)|  \\
                     &= |d(x, z) - d(y, w)|
\end{align*}
In the second step we used that $|a| + |b| \geq |a + b|$ for $a, b \in \mathbb{R}$.
\end{proof}

\newpage

\textbf{c)}
Let $(x_n)$ and $(y_n)$ be converging sequences in $X$ such that $\lim_{n
\rightarrow \infty} (x_n) = x$ and $\lim_{n \rightarrow \infty} (y_n) = y$.
Prove that $\lim_{n \rightarrow \infty} d(x_n, y_n) = d(x, y)$.

\newpage

%%%%%%%%%%%%%%%%%%%%%%%%%%%%%%%%%%%%%%%%%%%%%%%%%%%%%%%%%%%%

\textbf{Problem 2.}
Show that the limit of a convergent sequence in a metric space is unique.
\begin{proof}
Let $(x_n) \rightarrow x, (y_n) \rightarrow y$. Above we showed $d(x_n,
y_n) \rightarrow d(x, y)$. Take $y_n = x_n$. Then $d(x_n, y_n) = 0$ for all
$n$ and $d(x_n, y_n) \rightarrow 0$, implying $d(x, y) = 0$. Since we're
working in a metric space we have $x = y$.
\end{proof}

%%%%%%%%%%%%%%%%%%%%%%%%%%%%%%%%%%%%%%%%%%%%%%%%%%%%%%%%%%%%

\textbf{Problem 3.}
Let $(a_n)$ be a sequence in $\mathbb{R}$.

\textbf{a)}
There exists a subsequence of $(a_{n_k})_{k=1}^{\infty}$ of $(a_n)$
such that $\lim_k a_{n_k} = \lim \inf_n a_n$.
\begin{proof}
Suppose that $\lim \inf (a_n) = +\infty$. We construct a sequence $(b_n)$
diverging to $+ \infty$. Let $b_1 = a_1$. Let $b_2$ be the first element
$a_k$ in $a_n$ where $a_k > b_1 + 1$. Such an element must exist since the
terms in $a_n$ are not bounded above. Let $(\hat{a}_{n_k})$ be the new
subsequence of $(a_n)$ formed by only considering those terms in $(a_n)$
occuring after $a_k$. Repeat the process by choosing $b_3$ from the new
sequence $(\hat{a}_{n_k})$ where $b_3 > b_2 + 1$. Proceeding in this manner
we have that $(b_n)$ increases by at least 1 with every element, and hence
diverges to $+ \infty$. The case for $\lim \inf (a_n) = -\infty$ is
analogous; just choose $b_{i + 1} < b_{i} - 1$. 

Now suppose $\lim \inf a_n \rightarrow c$ for some $c \in \mathbb{R}$. We
construct a subsequence $(b_n)$ of $(a_n)$ with $(b_n) \rightarrow c$. Let
$b_1 = a_1$. Let $a_k$ be the first element in $(a_n)$ where $|a_k - c|
\leq |\frac{b_1 - c}{2}|$. Such an element exists since $\lim \inf a_n
\rightarrow c$ implies there are infinited terms in $(a_n)$ within
$\epsilon$
of $c$. Take $b_2 = a_k$. Again take the
subsequence of $(a_n)$ formed by only considering those terms in $(a_n)$
occuring after $a_k$. Repeat the process to create a sequence $(b_n)$ 
where $| b_n - c| \leq | \frac{b_1 - c}{2^n}|$. Hence $(b_n) \rightarrow c= \lim \inf_n a_n$.

\end{proof}

\textbf{b)}
$(a_n)$ converges to $a \in \mathbb{R}$ if and only if $\lim \inf_n a_n =
\lim \sup_n a_n = a$.

\newpage

%%%%%%%%%%%%%%%%%%%%%%%%%%%%%%%%%%%%%%%%%%%%%%%%%%%%%%%%%%%%

\textbf{Problem 4.}
Let $(X, d)$ be a metric space.

\textbf{a)}
The empty set and the set $X$ itself are both open and closed sets in $(X,
d).$
\begin{proof}
\end{proof}

\textbf{b)}
The intersection of a finite collection of open sets is open.
\begin{proof}
\end{proof}

\textbf{c)}
The union of an arbitrary collection of open sets is open.
\begin{proof}
\end{proof}

\textbf{d)}
The union of a finite collection of closed sets is closed.
\begin{proof}
\end{proof}

\textbf{e)}
The intersection of an arbitrary collection of closed sets is closed.
\begin{proof}
\end{proof}

%%%%%%%%%%%%%%%%%%%%%%%%%%%%%%%%%%%%%%%%%%%%%%%%%%%%%%%%%%%%

\textbf{Problem 5.}
Let $(X, d_X)$ and $(Y, d_Y)$ be metric spaces, $f: X \rightarrow Y$ a
continuous function and $B \subset Y$ a closed set. Then $A$ defined by
\[
    A = \{ x \in X | f(x) \in B \}
\]
is a closed set.
\begin{proof}
\end{proof}

%%%%%%%%%%%%%%%%%%%%%%%%%%%%%%%%%%%%%%%%%%%%%%%%%%%%%%%%%%%%

\textbf{Problem 6.}
Let $X$ be a Banach space and let $(x_n)$ be a sequence in $X$ such that
$\sum_{n=1}^{\infty} ||x_n|| = 1$.

\textbf{a)}
Prove that the series $\sum_{n=1}^{\infty} x_n$ also converges to a limit $x \in X$.
\begin{proof}
\end{proof}

\textbf{b)}
Prove that for an subsequence $(x_{n_k})_{k=1}^{\infty}$ of $(x_n)$, the 
series $\sum_{n=1}^{\infty} x_{n_k}$ also converges and that the norm of
its limit is bounded by 1.
\begin{proof}
\end{proof}


\end{document}

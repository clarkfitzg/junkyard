% A simple template for LaTeX documents

\documentclass[12pt]{article}

% Begin paragraphs with new line
\usepackage{parskip}  

% Change margin size
\usepackage[margin=1in]{geometry}   

% Supports proof environment
\usepackage{amsthm}

% Allows writing \implies and align*
\usepackage{amsmath}

% Allows mathbb{R}
\usepackage{amsfonts}

%%%%%%%%%%%%%%%%%%%%%%%%%%%%%%%%%%%%%%%%%%%%%%%%%%%%%%%%%%%%

\begin{document}

\section{Definitions}

\subsubsection{Metric Space}
A \textbf{metric} on $X$ is a function
\[
    d: X \times X \rightarrow \mathbb{R}
\]
such that

\begin{enumerate}
    \item Nonnegative $d(x, y) \geq 0$ for all $x, y \in X$.

    \item $d(x, y) = 0 \iff x = y$.

    \item Symmetric $d(x, y) = d(y, x)$.

    \item Triangle Inequality $d(x, y) \leq d(x, z) + d(z, y)$
\end{enumerate}

\subsubsection{Linear Space}
A linear space aka vector space defines a \textbf{norm} on $X$
\[
    || \cdot ||: X \rightarrow \mathbb{R}
\]
such that

\begin{enumerate}
    \item Nonnegative $||x|| \geq 0$ for all $x \in X$.
    \item $||x|| = 0 \iff x = 0$.
    \item Scalar multiplication $|| \lambda x || = |\lambda| \cdot ||x||$
    \item Triangle Inequality $||x + y|| \leq ||x|| + ||y||$

\end{enumerate}

There is a corresponding metric induced by the norm.


A sequence $(x_n)$ \textbf{converges} to $x \in X$ if for all $\epsilon > 0$ there
exists $N \in \mathbb{N}$ such that $d(x_n, x) < \epsilon$ for all $n \geq
N$.

A sequence $(x_n)$ is \textbf{Cauchy} if for all $\epsilon > 0$ there is $N
\in \mathbb{N}$ such that $d(x_m, x_n) < \epsilon$ for all $m, n \geq N$.

A metric space $(X, d)$ is \textbf{complete} if every Cauchy sequence in $X$
converges to a limit in $X$.

$G \subset X$ is \textbf{open} if for all $x \in G$ there exists $r > 0$
such that $B_r (x) \subset G$. $F \subset G$ is \textbf{closed} if the complement
$F^C$ is open.

$\bar{A}$, the \textbf{closure} of $A$ is the smallest closed set
containing $A$.

$K \subset X$ is \textbf{sequentially compact} if every sequence in $K$
contains a subsequence that converges in $K$.

$K \subset X$ is \textbf{compact} if every open cover of $K$ has a finite
subcover.

A subset of $X$ is \textbf{totally bounded} if there exists a finite
$\epsilon$ net for every $\epsilon > 0$.

$A \subset X$ is \textbf{dense} in $X$ if $\bar{A} = X$.

A subset of $X$ is \textbf{separable} if it has a countable dense subset.
Compact metric spaces are separable.

A subset of $X$ is \textbf{precompact} if its closure in $X$ is compact. 
$\iff$ Subset is totally bounded.


\subsubsection{Theorems}

A subset of a metric space is complete if and only if it is closed.

A subset of a metric space is compact if and only if it is complete and
totally bounded.

Continuous functions on a compact set are uniformly continuous.

The image of a compact set under a continous function is compact.

\subsection{Functions}

A function $f : X \rightarrow Y$ is \textbf{continuous} at $x_0 \in X$ if
for all $\epsilon > 0$ there exists $\delta > 0$ such that $d_X(x, x_0) <
\delta$ implies $d_Y(f(x), f(x_0)) < \epsilon$. Equivalently:
\begin{enumerate}
    \item There is a ball of positive radius $\delta$
    around a point $x_0$ that is mapped into the ball of radius $\epsilon$ around
    $f(x_0)$.
    \item $\lim_{x \rightarrow x_0} f(x) = f(x_0)$
\end{enumerate}

A function $f : X \rightarrow Y$ is \textbf{uniformly continuous} on $X$ if 
for all $\epsilon > 0$ there exists $\delta > 0$ such that $d_X(x, y) <
\delta$ implies $d_Y(f(x), f(y)) < \epsilon$ for all $x, y \in X$. 

The \textbf{sup norm} or $L^\infty$ norm between functions on $\mathbb{R}$
is defined as
\[
    d(f, g) = \sup_{x \in [0, 1]} |f(x) - g(x)|
\]

A sequence of bounded, real valued functions $(f_n)$ on a metric space
\textbf{converges uniformly} to a function $f$ if
\[
    \lim_{n =rightarrow \infty} || f_n -f || = 0
\]

$C(K)$, the space of continuous functions on a compact metric space $K$ is
complete.

\textbf{Weierstrass Approximation} The set of polynomials is dense in
$C([a, b])$.

\end{document}

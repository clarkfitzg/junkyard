% A simple template for LaTeX documents

\documentclass[12pt]{article}

% Begin paragraphs with new line
\usepackage{parskip}  

% Change margin size
\usepackage[margin=1in]{geometry}   

% Supports proof environment
\usepackage{amsthm}

% Allows writing \implies and align*
\usepackage{amsmath}

% Allows mathbb{R}
\usepackage{amsfonts}

%%%%%%%%%%%%%%%%%%%%%%%%%%%%%%%%%%%%%%%%%%%%%%%%%%%%%%%%%%%%

\begin{document}

\section{Definitions}

\subsubsection{Metric Space}
A \textbf{metric} on $X$ is a function
\[
    d: X \times X \rightarrow \mathbb{R}
\]
such that

\begin{enumerate}
    \item Nonnegative $d(x, y) \geq 0$ for all $x, y \in X$.

    \item $d(x, y) = 0 \iff x = y$.

    \item Symmetric $d(x, y) = d(y, x)$.

    \item Triangle Inequality $d(x, y) \leq d(x, z) + d(z, y)$
\end{enumerate}

A sequence $(x_n)$ \textbf{converges} to $x \in X$ if for all $\epsilon > 0$ there
exists $N \in \mathbb{N}$ such that $d(x_n, x) < \epsilon$ for all $n \geq
N$.

A sequence $(x_n)$ is \textbf{Cauchy} if for all $\epsilon > 0$ there is $N
\in \mathbb{N}$ such that $d(x_m, x_n) < \epsilon$ for all $m, n \geq N$.

A metric space $(X, d)$ is \textbf{complete} if every Cauchy sequence in $X$
converges to a limit in $X$.

$G \subset X$ is \textbf{open} if for all $x \in G$ there exists $r > 0$
such that $B_r (x) \subset G$. $F \subset G$ is \textbf{closed} if the complement
$F^C$ is open.

$K \subset X$ is \textbf{sequentially compact} if every sequence in $K$
contains a subsequence that converges in $K$.

$K \subset X$ is \textbf{compact} if every open cover of $K$ has a finite
subcover.

$A \subset X$ is \textbf{totally bounded} if there exists a finite
$\epsilon$ net for every $\epsilon > 0$.

\subsubsection{Theorems}

A subset of a metric space is complete if and only if it is closed.

A subset of a metric space is compact if and only if it is complete and
totally bounded.

\subsection{Functions}

A function $f : X \rightarrow Y$ is \textbf{continuous} at $x_0 \in X$ if
for all $\epsilon > 0$ there exists $\delta > 0$ such that $d_X(x, x_0) <
\delta$ implies $d_Y(f(x), f(x_0)) < \epsilon$. Equivalently:
\begin{enumerate}
    \item There is a ball of positive radius $\delta$
    around a point $x_0$ that is mapped into the ball of radius $\epsilon$ around
    $f(x_0)$.
    \item $\lim_{x \rightarrow x_0} f(x) = f(x_0)$
\end{enumerate}

A function $f : X \rightarrow Y$ is \textbf{uniformly continuous} on $X$ if 
for all $\epsilon > 0$ there exists $\delta > 0$ such that $d_X(x, y) <
\delta$ implies $d_Y(f(x), f(y)) < \epsilon$ for all $x, y \in X$. 



\end{document}

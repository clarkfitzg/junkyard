\documentclass[11pt]{article}

% Change margin size
\usepackage[margin=1in]{geometry}   

% Don't number subsections
\setcounter{secnumdepth}{2}

\usepackage{setspace}
\doublespacing

\begin{document}

This is an abstract for the first chapter of my thesis. This morning I
revised it to improve the motivation for a general audience, and to improve
the style. I'm not confident the revised version is ideal.

\subsubsection{Original}

In this chapter we describe how to statically analyze R code to construct a
data structure representing the ideal parallelism implicit in the code. It
shows the ideal parallelism by capturing all the constraints on execution
and identifying map type operations. It represents several types of data
dependencies between statements, so it is a task graph. It goes beyond a
task graph in a couple ways. First, it identifies vectorized and data
parallel statements, which allows us to fuse implicit loops to improve data
locality during parallel execution. Second, it represents process
dependencies essential for the correct execution of code, for example
plotting commands which must run in order on the same process.

\subsubsection{Revised}

Programs express parallelism through data and tasks. Data parallel code
executes the \emph{same} instructions across many data simultaneously. Task
parallel code executes two or more \emph{different} instructions
simultaneously. Our ultimate goal is to create a more efficient parallel
program from one that is not parallel, ie. serial. We need to know where a
serial program can use data or task parallelism.

The first chapter describes how to infer parallelism implicit in R code. A
program statically analyzes the code to construct a data structure that
represents constraints on expression ordering. The data structure
represents several types of data dependencies between statements, so it is
a task graph that shows all possible ways to group code into tasks and run
in parallel. It goes beyond a task graph in two ways. First, it identifies
vectorized and apply style functions as data parallel. Second, it
represents process dependencies essential for the correct execution of
code, for example plotting commands which have no data depenedencies yet
must run in order on the same process.

\end{document}
